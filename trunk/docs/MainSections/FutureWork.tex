\section{Future Work}

This application has a great potential for future work. This section details some of the ways in which the application could be extended/improved:

\subsection{Better Calendar Integration}
This system currently uses Google Calendar, in the future it would be useful to extend this to use different calendar such as university timetables so that certain events such as lectures don't need to be input manually to Google Calendar. 

\subsection{Adding Favourite Locations}
Adding the notion of favourite products and locations would enable the user to customise the caffeine suggestions that they recieve from the application. For instance if the user only liked tea and coffee but not energy drinks then that would be taken into account for the predictions and they might be advised to consume caffeine more regularly as they favour drinks with a lesser caffeine content. 

\subsection{Accurate Caffeine Levels Per Product}
For the purposes of the prototype Opticaff only used average values for the four categories of caffeinated beverages: (coffee, tea, soft drinks and energy drinks) and assumed an average size for each beverage. For a final application each listed product would have it's appropriate caffeine content listed. 

\subsection{Advanced Leaderboard Functionality}
Currently within the prototype there is a leaderboard for the optimum caffeine levels. In the finished application it would add a better competative edge to add additional functionality such as a history of scores, mapping caffeine levels to a specific event (e.g I was at an optimum caffeine level during this lecture). It would also encourage competativeness if there was the notion of adding friends so that users could compete directly against their peers. There could also be the notion of groups such as course groups or module groups to see who maintained the optimum caffeine levels within a specific group. 

\subsection{Web Interface}
Having a web interface that users could also use to update their information or to view leaderboard statistics could increase the amount of people using Opticaff as there would be multiple ways they could use it. 

\subsection{Dynamic Stock Reflection}
If caffeine points of service were to start to release (either publicly or privately just to Opticaff) stock levels of products dynamically then Opticaff could take this information into account and only direct users to places that had the desired beverage in stock. 
 
\subsection{Health Warnings}
Despite its useful traits, Caffeine is still a drug and like all drugs needs to be managed carefully. Opticaff aims to aid the users in managing their caffeine content and keeping it at an optimum level. However, it would be worth adding health warnings about the risks of overdosing on caffeine and to make it clear that Opticaff promotes responsible useage both for health and legal reasons. 

\subsection{Adaptation to other Universities}
Given that this system uses Google Calendar to verify the users daily activity, and that the system has been built to import data from a set of SPARQL about the caffeine locations; adapting this application for multiple universities wouldn't take very long. The users would continue to add their timetable data in as before, and a new set of queries would be built for that university. This would enable Opticaff to be pitched to various university establishments for minimum additional development time. 

\subsection{Adaption to Specific Coffee Chains}
Based on similar principles as the idea above, this application could be adapted to a specific Coffee Chain such as Costa or Starbucks if they gave Opticaff access to their location and product data. This would then enable the app to be used by anyone who was a fan of caffeine or indeed these specific stores as large chains such as these have branches all over the country.
