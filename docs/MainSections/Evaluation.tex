\section{Evaluation}

\subsection{Product Evaluation}
In regards to achieveing the projects aims, four out of the five goals set out in the initial planning stages were completely met, and one of them was partially met. The table belows details this:

\begin{tabular}{|p{20pt}|p{140pt}|p{40pt}|p{210pt}|}
\hline
\textbf{No} & \textbf{Requirement} 													& \textbf{Met} 	& \textbf{Description} 
\\\hline
1		& Obtain data regarding caffeine sources in and around the University of Southampton.					& Yes		& All points of service that are listed as providing caffeine are used within our system. 
\\\hline
2 		& Store University Timetable data using Sussed										& Partially	& As discussed earlier in section \ref{sec:calendar} accessing the timetable data from Sussed wasn't a simple task so Google Calendar was used instead. This does mean that the users have to input their own timetable data, however it does meet the requirement of using calendar data. 
\\\hline
3 		& Make calculations about user caffeine intake and make predictions for the optimum time to consume next.		& Yes		& Opticaff works out the caffeine decay rate (based on the calculations specified in section \ref{sec:decay}) and based on the users timetabled events makes a prediction for the best time to consume the next caffeinated beverage. 	
\\\hline
4 		& Have the notion of a leaderboard to rank users ``productvity" based on maintaining their optimum caffeine levels.	& Yes		& A basic leaderboard has been implemented that ranks users based on their caffeine consumption. 
\\\hline
5 		& Ascertain the users position and locate the closest appropriate caffeine sources to them.				& Yes		& The long/lat values of all the caffeine points of service are stored by Opticaff. By using GPS the users location can be determined also and then they can be directed to the appropriate destination. 
\\\hline
\end{tabular}

There were a few problems that the group faced along the way, the main one apart from the timetable one was in regards to storing the data itself. The initial attempt was to store the data regarding the caffeine points of service on the phone itself to avoid numerous network operations. However it became apparent that to manage it effectively and to maintain the leaderboard functionality it needed to be stored externally from the mobile devices, thus Google App Engine was used to store it.  

Overall however the group feels that the project was very successfull in meeting its aims and all members were happy with the final product and the features it implemented.

\subsection{Team Evaluation}
The group feel that they worked well together as a team. Everyone participated and fulfilled the two roles they were assigned at the beginning of the project. The Bi-weekly meetings were kept to with additional sessions when necessary. The project met its requirements and was delivered on time for the deadline. 

