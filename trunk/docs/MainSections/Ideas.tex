\section{Project Conceptualisation \& Monetisation}

This section details an outline of the ideas that were conceived and the justifications for the final idea coupled with its monetisation potential. 

\subsection{Ideas}

It was decided from the early stages that the application would use the University of Southamptons open linked data (see section \ref{sec:Data} relating to its internal organisations. The focus of this data would be the points of service (e.g cafes, vending machines) that sold caffeine in and around the University. After establishing the data sets a number of ideas of how to best uitlise this data were brought to the forefront. In order to establish the most valuable idea, each potential solution was weighed against certain aspects: 

\begin{itemize}
	\item{\textbf{Uniqueness Factor:} Was it a new novel idea and if so how?}
	\item{\textbf{Monetisation Potential:} Was there the potential to make money out of it?}
	\item{\textbf{User Interest:} Would it capture users on a long term basis as opposed to just on a novelty?}
\end{itemize}

\subsubsection{Caffeine Finder}
An initial idea was a caffeine finder application that allowed the user to find the nearest, available, appropriate (e.g Student or Staff) caffeine to them at any given time. 

\begin{itemize}
	\item{\textbf{Uniqueness Factor:} As stated in the competitors research section (section \ref{sec:competitors} there are apps similar to this in the market currently. It still has a slight unique factor in that it includes all of the University affiliated points of service including vending machines as well as nearby external stores. In addition it also covers a range of caffeinated products as opposed to focusing on tea and coffee. Nonetheless there are still similar applications out that perform similar functions.}
	\item{\textbf{Monetisation Potential:} There would be the potential to ask specific caffeine selling stores to invest in return for putting them at the top of the applications recommendations.}
	\item{\textbf{User Interest:} This idea doesn't have a great potential for capturing users interest on a long term basis. It has a novelty factor of showing the users where and when they can purchase caffeine, but runs the risk of loosing interest after the users have used it enough to retain any useful information.}
\end{itemize}

\subsubsection{Justification for Rejection}
Overall this application was considered a decent idea. It wasn't greatly unique but still appealed to both a wide and specialised market in that it would be of use to everyone who frequented the University and it covered all the points of service in a nearby range. However, its main downfall was its lack of ability to maintain a user interest, as without that potential it would be infeasible to expect external parties to invest in it. 

\subsubsection{Caffeine Notifications}
Another idea was to make a very simple application that essentially buzzed and notified the user whenever they walked past somewhere that sold caffeine.

\begin{itemize}
	\item{\textbf{Uniqueness Factor:} This was debatably more unique than a finder application as it would specifically alert the user to the presence of caffeine as opposed to waiting for them to search for it.}
	\item{\textbf{Monetisation Potential:} This idea didn't have a great deal of monetisation potential. If it's going to buzz whenever it passes a caffeine location then there was no feasible way to promote specific locations over others. The only way this could have been monetised would have been to charge for the app. However despite it's potential usefulness it still seemed unlikely that users would actually pay for the service when realistically they could just pay attention and get to know the locations and achieve the same result.}
	\item{\textbf{User Interest:} This application could easily hold user interest as its a very simple concept and doesn't require any effort on their part apart from keeping the app running in the background.}
\end{itemize}

\subsubsection{Justification for Rejection}
This application had benefits in its simplicity but with its lack of monetisation potential it was unsuitable for this project. 

\subsubsection{Caffeine Productivity with a Competative Edge - Opticaff}
The final idea expanded on our inital idea of a caffeine finder application but with an additional function to tie the user interest in. Opticaff is an application that allows users to input their daily timetable and based on user input of caffeine consumption and personal details, calculates when they need to consume their next caffeinated beverage. It uses your calendar for the day to determine if you have any events left (e.g a lecture) and if so it calculates when you should consume caffeine and what strength it should be to maintain optimum caffeine levels throughout. However, this is not the only functionality of the application. It also has the concept of competing with your peers to maintain the optimum caffeine levels with a leaderboard to showcase the most productive users. 

\begin{itemize}
	\item{\textbf{Uniqueness Factor:} This application had a unique factor in that it had more functionality than any of the competitors mentioned in section \ref{sec:competitors}}. There are applications in the market that monitor caffeine consumption, applications that locate caffeine, and competative applications. This application combines all three for a truly unique product. 
	\item{\textbf{Monetisation Potential:} This idea retained the potential to be promoted to owners of the caffeine points of service for investment. It could also potentially be pitched to be sponsered in a research function to investigate the effects of monitoring caffeine intake on productivity.}
	\item{\textbf{User Interest:} This idea has by far the best user interest potential out of the three. It's not just a novelty application as people consume caffeine every day, and based on their different calendars (which for uni students and lecturers will differ by day) and external factors such as amount of sleep the night before their caffeine consumption will differ each day. }
\end{itemize}

\subsubsection{Justification for Acceptance}
Opticaff was chosen as the final idea as it fulfilled all three of the criterion listed above, and fulfilled all of them in a superior fashion. It has the unique factor not through its individual ideas but through the combination of these into a multifunctional app that not only allows you to monitor your caffeine consumption, but to locate places that sell caffeine and tie the two together to optimise your caffeine intake. In addition to this it has the competative element which aids in capturing and retaining user interest as well as boosting potential monetisation potential. 

\subsection{Monetisation}
This application has a huge monetisation potential. The obvious initial step would be to pitch for funding from the owners of the caffeine points of service, however there is scope for it to go much further. 

Competative applications are becoming increasingly popular as has been shown by our research (INSERT REF HERE). If the application became popular then advertising space could be sold, and two copies of the application could be made available: a free one with adverts and a paid for one without which has been shown as a viable strategy to ``convert" people to the paid app \cite{adverts} \cite{drawsomething}.