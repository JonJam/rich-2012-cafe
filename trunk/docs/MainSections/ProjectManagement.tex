\section{Project Management \& Tools}

\subsection{Project Management}
This section details how this project was managed, the roles of the different group members and the methodologies that were used.

\subsubsection{Team Roles}
The following roles were outlined for this project:

\begin{itemize}
	\item{\textbf{O}rganiser - Oversees project and time management}
	\item{\textbf{D}eveloper - Develops the application}
	\item{\textbf{U}sability Expert - Manages User Interface and HCI (Human Computer Interaction)}
	\item{\textbf{R}esearcher - Researches background information}
	\item{\textbf{P}resenter - Presents pitch \& manages presentation}
\end{itemize}

This table details the roles each member of the team undertook:

\begin{tabular}{|p{90pt}|p{30pt}|p{300pt}|}
\hline
\textbf{Team Member} 	& \textbf{Roles} 	& \textbf{Description} 
\\\hline
Adam Costello 		& \textbf{R, P} 	& Adam coordinated with Craig on the Presentation section of this project, and worked with Sami on the research section focusing on the market research. 
\\\hline 
Mike Elkins 		& \textbf{U, D} 	& Mike coordinated with Pratik to design and build the User Interface section of the application.
\\\hline
Jonathan Harrison 	& \textbf{O, D} 	& Jonathan was in charge of organising the team and making sure tasks were completed in a timely manner. He also worked with Sami on the backend element of the							  application concentrating on the calendar and scheduling the system to import the necessary data.
\\\hline
Sami Kanza 		& \textbf{R, D} 	& Sami coordinated with Jonathan on the backend of the application, concentrating on the queries to retrieve the data, she also worked with Adam on the research						  section, focusing on the caffeine research.
\\\hline 
Pratik Patel 		& \textbf{U, D} 	& Pratik worked with Mike to design and build the User Interface section of the application. 
\\\hline
Craig Saunders 		& \textbf{P, D} 	& Craig worked on the location and directing element of the application in addition to working with Adam on the Presentation section. 
\\\hline
\end{tabular}

\subsubsection{Team Organisation}
The team decided that bi-weekly meetings would be appropriate for this project. In each meeting the progress made between then and the previous meeting was stated, any problems that needed solving were raised and then the tasks for the next meeting were divided up. The nature of these meetings were similar to a SCRUM \cite{scrum} daily standup, but were obviously performed less frequently than that.  

\subsection{Group Methodologies}
The team decided to implement several agile-based technologies to aid with the application development process. The development areas have been broken down into individual tasks (stories) to simplify the process. An iterative development process was also adopted, with the base of the application built first, followed by a gradual development of its features. 

Pair Programming was also an agile technique that the team used. The application development was split into three areas: Database development and SPARQL querying, the map interface and GPS positioning / directions, and the overall user interface. Each section had two group members assigned to it, and each pair worked together to share their skillset and therefore produce a superior result. 

\subsection{Tools \& Techniques}
This section details the tools used by the team to aid with their project management and development.

\subsubsection{Version Control}
\label{sec:VersionControl}
There were several version control mechanisms we could have used. Git and SVN were both considered, with potential to store the code on UGForge or Googlecode or Github etc. In the end the group chose to use SVN Googlecode. SVN was chosen primarily due to the fact that Eclipse was being used as the main development environment, and there is an SVN plugin for Eclipse that makes the subversioning process easier to integrate with the development process. Googlecode was chosen because the application that is being developed is for Android, and therefore it was felt that Android’s creators Google would be the most sensible place to store the application. 

\subsubsection{Data Source}
OptiCaff used the Open Data Service from the University of Southampton \cite{DataSouthampton} to retrieve the relevant information for the application. This service provides open linked data about some of the administrative information regarding the university. It also provides a SPARQL Endpoint \cite{SotonSparql} (a service which facilitates users querying a knowledge base using the SPARQL query language) \cite{SparqlEndpoint}. OptiCaff utilises this with a few specialist queries, and combined with user preferences can provide the user with a wide selection of caffeine choices around campus. 

\subsubsection{Development Tools}
Given that OptiCaff was to be an Android application. The development tools needed to facilitate its production were the Android SDK tools \cite{AndroidSDK}. The recommended development environment suggested by Android was to use Eclipse \cite{Eclipse} with the ADT (Android Developer Tools) plugin \cite{SDKAllOS}. Given the groups overall familiarity with Eclipse and it’s additional useful plugins for version control (see section \ref{sec:VersionControl}) this was used as our IDE of choice.

\subsubsection{Communication Tools}
The Team used Google Docs to share documentation and task lists, and Facebook to communicate via group chat. 
