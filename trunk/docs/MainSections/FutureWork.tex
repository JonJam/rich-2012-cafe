\section{Future Work}

OptiCaff has a great potential for future work. This section details some of the ways in which it could be extended/improved:

\begin{itemize}
	\item{\textbf{Improved Calendar Integration:}
		This system currently uses calendars synchronized to a users phone, it would be useful to extend this to include  		    university timetables so that events such as lectures don't need to be manually input.
	}
	\item{\textbf{Adding Favourite Locations:}
		Adding favourite products and locations would enable the user to customise the caffeine 
		suggestions provided by the application. For example if the user only liked tea and coffee but not energy
		drinks then that would be taken into account and they might be advised to consume
		caffeine more regularly as they favour drinks with a lesser caffeine content. 
	}
	\item{\textbf{Accurate Caffeine Levels:}
		For the purposes of the prototype Opticaff only used average values for the four categories of caffeinated
		beverages (coffee, tea, soft drinks and energy drinks) and assumed an average size for each beverage. An
		improved OptiCaff would list each product's correct caffeine content.
	}
	\item{\textbf{Advanced Leaderboard Functionality:}
	Currently within the prototype there is a leaderboard for the optimum caffeine levels. The finished application
	would increase the competitive edge by adding: history of scores, mapping caffeine levels to events (e.g I was the
	most productive in this lecture) and adding friends/groups so that users could compete directly with people in
	similar circumstances.
	}
	\item{\textbf{Adaptation to other Universities:}
	Given that this system uses Android Calendar to verify the users daily activity, and that the system has been
	built to import data from a set of SPARQL about the caffeine locations; adapting this application for multiple
	universities would not take very long. The users would continue to add their timetable data in as before, and a
	new set of queries would be built for that university. This would enable OptiCaff to be pitched to various
	university establishments for minimum additional development time.  
	}
	\item{\textbf{Adaption to Specific Coffee Chains:}
	Based on similar principles as the idea above, this application could be adapted to a specific coffee chain such
	as Costa or Starbucks if they gave Opticaff access to their location and product data. This would then enable the 
	app to be used by anyone who was a fan of caffeine or indeed these specific stores as large chains such as these 
	have branches all over the country.
	}
\end{itemize}