\section{Introduction}

OptiCaff is the new innovative way to maximise your productivity during university hours! By analysing your caffeine content in addition to your weekly timetable, OptiCaff can work out the optimum time for your next caffeine consumption. However, this is not all it can do. Not only will it notify you when its time for you to consume your next caffeinated beverage, it will tell you what type will be most effective (tea, coffee, energy drinks) and then point you to the nearest place in and around the University of Southampton that sells that product. It also has a competative element to it, users can compete with one another to appear on the leaderboard listing the most productive users. 

\subsection{Project Problem}
Modern life is both busy and stressfull. People have long since started turning to caffeine to help them to manage their tiredness levels and keep them going throughout the day. However, caffeine despite its uses is still a drug and can have adverse effects if misused. In addition to this its not always easy to remember where the closest point of service to you is or whether they sell the type of caffeine you are after. Opticaff aims to solve these problems by aiding the user in managing their caffeine dosage in addition to pointing them towards the closest place to buy their desired beverage. 

\subsection{Project Audience}
The current audience of this project is anyone that visits the University of Southampton or surrounding area on a regular basis. The points of service that are pointed to by our applications are all either University of Southampton affiliated or very near the main campus and therefore are regularly accessible to any frequent visitors of the University. 

\subsection{Project Goals}
The goals of this project are to produce a prototype that has the following functionality: 

\begin{enumerate}
	\item{Obtain data regarding caffeine sources in and around the University of Southampton.}
	\item{Store University Timetable data using Sussed}
	\item{Make calculations about user caffeine intake and make predictions for the optimum time to consume next.}
	\item{Have the notion of a leaderboard to rank users ``productvity" based on maintaining their optimum caffeine levels.}
	\item{Ascertain the users position and locate the closest appropriate caffeine sources to them.}
\end{enumerate}

\subsection{Project Scope}
This is a prototype application to showcase the main features and to illustrate what the full application will look like. The prototype produced will display all the features available in the user interface and the key ones such as: consuming caffeine, calculations based on your timetable, ranking on the leaderboard etc will be implemented. This application will use researched averages for the caffeine content of drinks as opposed to storing individual data for each one and will make assumptions for the average man and woman in terms of optimum caffeine intake. In addition the calendar the application uses to base the users daily events off will be statically added for the purposes of this prototype. 
