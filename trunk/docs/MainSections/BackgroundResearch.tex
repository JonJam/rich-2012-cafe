\section{Background Research \& Analysis}

\subsection{Market Research}

\subsubsection{Coffee Research}
Coffee consumption in the United Kingdom has steadily increased over the past decade. In particular the past five years been seen an explosive increase, there are several theories as to why this is the case.

Firstly, instant coffee shops have become more common on our high streets. Companies such as Starbucks and Costa have been opening more stores as more Brits have been buying instant coffee, this doesn’t show any signs of slowing down either as Starbucks have recently announced 300 new stores to be opened over the next five years \cite{starbucks}.

Secondly, these brands have contributed to the newfound ‘coolness’ that is associated with coffee. 

Lastly, there is evidence that the economic climate has played a part in coffee’s rise. Also known as the ‘lipstick effect’, Britons have been unable to afford expensive treats for themselves so they have been spending on cheaper treats, a good example of which is coffee \cite{costa}.

This research is important to OptiCaff because it shows that the coffee industry is on the rise, it would not make business sense for us to invest in a declining industry.

PUT IN COFFEE GRAPH HERE 

\subsubsection{Competative Apps}

\subsection{Mobile Platform Research}
There are two mobile operating systems that OptiCaff could be deployed upon. The Android operating system created by Google requires applications to be written in Java which is a positive for the OptiCaff team, all members have Java development experience with one member having Android development experience. Using Android also means no special hardware is required to develop an application, the Android SDK is freely available and IDE’s are also free, Eclipse being an example of this \cite{Eclipse}. The app store for Android phones, Google Play, is a large marketplace that would give OptiCaff a great enough set of potential users.

``Google’s 10 Billion Android App Downloads: By the Numbers" \newline
A negative of Android is the result of it being open-source, manufacturers are able to create a separate version of Android just for their hardware, this means there are now lots of iterations of Android. Developing an application for all these different versions is difficult and OptiCaff would not be compatible with all phones at first.

iOS is the Apple iPhone’s operating system, it requires applications to be written in Objective C, no members in the OptiCaff team have any experience with Objective C. The only IDE available for developing iPhone applications is Xcode which is only available to OS X users, in other words, development is only possible on Apple computers, this is a requirement that is particularly limiting for a small development team. The Apple app store is currently the largest marketplace for mobile applications, although it is slightly ahead of Google Play. Finally, there is only one iteration of iOS, it is not licensed to other companies, the only complication is that there are a few versions running concurrently. This means that an app developed for the iPhone would likely work on all iPhones.

OptiCaff will be developed in Android, the reasons for this are that the members of the team are more comfortable with Java development and not all members of the team have an Apple computer and therefore would not be able to aid in development.

\subsection{Competitors Research}
\label{sec:Competitors}

\subsection{Caffeine Research}
\label{sec:Caffeine}
Given the vast range of different cafffeinated products and the limited time to produce a prototype application, it was decided that the products displayed by OptiCaff would be grouped into four different types of drink, and each type would be allocated an average caffeine content. Below is a table showing these totals, which were got from these sources \cite{Coke} \cite{TeaCoffee} \cite{EnergyDrink}.

NEED CAFFEINE DECAY IN HERE

\begin{center}
\begin{tabular}{|l|l|}
\hline
\textbf{Drink Category} & \textbf{Average Caffeine Content (mg)} \\\hline
Tea & 40 \\\hline
Coffee & 54 \\\hline
Energy Drinks & 80 \\\hline
Soft Drinks & 34.5 \\\hline
\end{tabular}
\end{center}

\subsection{Summary}
decisions \& why 