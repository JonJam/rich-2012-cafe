\section{Introduction}
OptiCaff is a revolutionary way of managing a user's caffeine level whilst respecting and reacting to their daily activities.
OptiCaff is a smartphone application built for Android. 
Features include:
Tracking the user's caffeine levels,
suggesting the optimal time for caffeine consumption to fit around their schedule,
locating the nearest caffeine vendor,
and detailing the recommended beverage options provided.
OptiCaff also features a competitive element where users compete with each other to appear on the leaderboard listing the most optimal caffeine consumers. 

\subsection{Project Problem}
In this modern age, peoples lives have become increasingly active, making tiredness a likely side effect. 
Consuming caffeine has become a popular way of managing this for many people \cite{Popularity}.
Despite its usefulness, the fact remains that caffeine is still a drug and can have adverse effects if misused \cite{misuse}.
Using caffeine effectively is a difficult skill to master, requiring logging the caffeine content of each beverage in addition to consumption time and then performing complex calculations to ascertain the optimum time for the next caffeinated beverage.
OptiCaff aims to solve this problem by monitoring a users caffeine intake in addition to their timetabled activities so that consumption suggestions work in harmony with the users routine.
In addition to this it also points the user to the closest place to purchase their desired beverage to minimise their detour.  
  
\subsection{Project Audience}
The initial target audience of OptiCaff are people who visit the University of Southampton Campus on a regular basis. 

\subsection{Project Goals}
The goals of this project are to produce a prototype with the following functionality: 

\begin{enumerate}
	\item{Obtain and use linked data detailing caffeine sources within the University of Southampton.}
	\item{Integrate users schedule}
	\item{Monitor users caffeine intake and predict the optimum time for their next consumption.}
	\item{Provide a leaderboard to rank users based on how well they maintain their caffeine level within the optimum range during their scheduled events.}
	\item{Approximate the users position and locate the closest appropriate caffeine sources to them.}
\end{enumerate}

\subsection{Project Scope}
This is a prototype application to showcase the main features and to illustrate what the full application will look like. 
The prototype produced will implement the following key features of consuming caffeine, performing calculations based on a users timetable, give directions to the nearest appropriate caffeine source, and ranking on the leaderboard. 
This application will use researched averages for the caffeine content of drinks as opposed to storing individual data for each one and will make assumptions for the average man and woman in terms of optimum caffeine intake.  
