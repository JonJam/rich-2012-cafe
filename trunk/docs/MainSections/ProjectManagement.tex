\section{Project Management \& Tools}

\subsection{Project Management}
This section details how the project was managed, the roles of the different group members and the methodologies that were used.

\subsubsection{Team Roles}
The following roles were outlined for this project and assigments are shown in table \ref{tab:Roles}:

\begin{itemize}[noitemsep,leftmargin=1cm]
	\item{\textbf{O}rganiser - Oversees project and time management}
	\item{\textbf{D}eveloper - Develops the application}
	\item{\textbf{U}sability Expert - Manages User Interface and HCI (Human Computer Interaction)}
	\item{\textbf{R}esearcher - Researches background information}
	\item{\textbf{P}resenter - Presents pitch \& manages presentation}
\end{itemize}

\begin{table}[ht]
\caption{Table detailing Team Roles}
\label{tab:Roles}
\begin{tabular}{|p{90pt}|p{30pt}|p{304pt}|}
\hline
\textbf{Team Member} 	& \textbf{Roles} 	& \textbf{Description} 
\\\hline
Adam Costello 		& \textbf{R, P} 	& Adam coordinated with Craig on the Presentation, and worked with Sami on documentation and research. 
\\\hline 
Mike Elkins 		& \textbf{U, D} 	& Mike coordinated with Pratik to design and build the User Interface section of the application along with other sections.
\\\hline
Jonathan Harrison 	& \textbf{O, D} 	& Jonathan organised the team and made sure tasks were completed in a timely manner. He also worked with Sami on the backend element of the							  application.
\\\hline
Sami Kanza 		& \textbf{R, D} 	& Sami coordinated with Jonathan on the backend of the application, she also worked with Adam on the documentation and research						  section, focusing on the caffeine research.
\\\hline 
Pratik Patel 		& \textbf{U, D} 	& Pratik worked with Mike to design and build the User Interface section of the application along with other sections. 
\\\hline
Craig Saunders 		& \textbf{P, D} 	& Craig worked on the core app functionality in addition to working with Adam on the Presentation section. 
\\\hline
\end{tabular}
\end{table}

\subsubsection{Team Organisation}
The team decided that bi-weekly meetings would be appropriate for the project. In each meeting the progress made between then and the previous meeting was stated, any problems that needed solving were raised and then the tasks for the next meeting were divided up. The nature of these meetings were similar to a SCRUM \cite{scrum} daily standup.  

\subsection{Group Methodologies}
The team decided to implement several agile-based techniques to aid with the application development process. The development areas have been broken down into individual tasks (stories) to simplify the process. An iterative development process was also adopted, with the base of the application built first, followed by a gradual development of its features. 

Pair Programming is another agile technique that the team used. The application development was initially split into three areas: Database development/SPARQL querying, map interface and GPS positioning/directions, and the application UI. Each section had two team members assigned to it, and each pair worked together to combine skillsets, therefore producing superior results. 

\subsection{Tools \& Techniques}
This section details the tools used by the team to aid with project management and development.

\subsubsection{Version Control}
\label{sec:VersionControl}
There were several version control mechanisms we could have used. Git and SVN were both considered, with potential to store the code on UGForge,Googlecode or Github etc. In the end the group chose to use SVN Googlecode. SVN was chosen primarily due to the fact that all team members had experience with it. Googlecode was chosen because the application that is being developed is for Android, and therefore it was felt that Android’s creators (Google) would be the most sensible place to store the application. 

\subsubsection{Development Tools}
The development tools used in OptiCaff's production were the Android SDK tools \cite{AndroidSDK}. The recommended development environment suggested by Android was Eclipse \cite{Eclipse} accompanied by the ADT (Android Developer Tools) plugin \cite{SDKAllOS}. Given the groups overall familiarity with Eclipse and it’s additional useful plugins for version control (see section \ref{sec:VersionControl}) this was used as our IDE of choice.