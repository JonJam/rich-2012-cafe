\section{Project Conceptualisation \& Monetisation}

This section details the ideas that were conceived and the justifications for the final idea coupled with its monetisation potential. 

\subsection{Acceptance Criteria}
It was decided from the early stages that the application would use the University of Southamptons open linked data (see section \ref{sec:Data} relating to its internal organisations). 
The focus of this data would be the points of service (e.g cafes, vending machines) that sell caffeine in and around the University. 
After establishing the data sets a number of ideas of how to best utilise this data were discussed. 
In order to establish the most valuable idea, each potential solution was weighed against certain aspects: 

\begin{itemize}
	\item{\textbf{Uniqueness Factor:} Is it a new novel idea and if so how?}
	\item{\textbf{Monetisation Potential:} Is there the potential to monetise it?}
	\item{\textbf{User Interest:} Will it retain users on a long term basis as opposed to just on a novelty?}
\end{itemize}

\subsection{Ideas}
The three ideas that were contemplated are listed below. Table \ref{tab:Ideas}details their weighting against the acceptance criteria:

\begin{itemize}
	\item{\textbf{Caffeine Finder:} Allows users to find the nearest available caffeine vendors to them at any given time.}
	\item{\textbf{Caffeine Notify:} Buzzes and notifies the user when they pass caffeine vendors.}
	\item{\textbf{OptiCaff:} Tracks users caffeine levels, and suggests and locates caffeine vendors at an appropriate time for the next caffeinated beverage. Incorporates a leaderboard for users to compete with one another over maintaining optimum caffeine levels.}
\end{itemize}

\begin{table}[ht]
\caption{Table of Potential Ideas}
\label{tab:Ideas}
\begin{tabular}{|p{45pt}|p{102pt}|p{104pt}|p{95pt}|p{10pt} |p{10pt} |p{10pt}|}
\hline
	Idea 
	& Uniqueness (U)
	& Monetisation (M)
	& User Interest (I)
	& U
	& M
	& I
\\\hline
	Caffeine Finder 
	& Includes University vendors and vending machines. 		
	& Pitchable to caffeine vendors and can advertise within.
	& Easy to retain the information, so no need for reuse. 
	& \LARGE{\textcolor{green}{\Pisymbol {pzd} {52}}}
	& \LARGE{\textcolor{green}{\Pisymbol {pzd} {52}}}
	& \LARGE{\textcolor{red}{\Pisymbol {pzd} {56}}}
\\\hline
	Caffeine Notify 
	& Alerts the user to caffeine rather than them searching for it.
	& No benefit to vendors, also users unlikely to pay for it.
	& Easy to leave on in the background. 
	& \LARGE{\textcolor{green}{\Pisymbol {pzd} {52}}}
	& \LARGE{\textcolor{red}{\Pisymbol {pzd} {56}}} 
	& \LARGE{\textcolor{green}{\Pisymbol {pzd} {52}}}
\\\hline
	OptiCaff 
	& Combines different elements to provide a new level of service.
	& Pitchable to caffeine vendors and can advertise within.
	& Schedules/Caffeine levels differ daily, therefore reusable. 
	& \LARGE{\textcolor{green}{\Pisymbol {pzd} {52}}}
	& \LARGE{\textcolor{green}{\Pisymbol {pzd} {52}}}
	& \LARGE{\textcolor{green}{\Pisymbol {pzd} {52}}} 
\\\hline
\end{tabular}
\end{table}
\vspace{0.5cm}
\subsection{Final Choice - OptiCaff}
OptiCaff was chosen as the final idea as it fulfilled all three of the criterion listed above.
It has the unique factor not through its individual ideas but through the combination of these into a multifunctional app that not only allows you to monitor your caffeine consumption, but to locate places that sell caffeine and tie the two together to optimise your caffeine consumption. 
In addition to this it has the competative element which aids in capturing and retaining user interest as well as boosting monetising potential.

