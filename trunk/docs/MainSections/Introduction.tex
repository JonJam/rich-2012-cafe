\section{Introduction}

OptiCaff is the new innovative way to maximise productivity during university hours! 
Opticaff can calculate the optimum times to consume caffeine by analysing a users caffeine content and their timetable for the day. 
This is however not all it can do; not only will it notify a user when its time for them to consume their next caffeinated beverage, it will suggest what type of drink would be most effective (tea, coffee, energy drinks). 
It will then inform the user of the nearest place in the University of Southampton that sells that product.
Opticaff also features a competative element where users can compete with each other to appear on the leaderboard listing the most productive users.  

\subsection{Project Problem}
Modern life is both busy and stressful and consuming caffeine has become a popular way of managing this for many people \cite{Popularity}.
Despite caffeines usefulness, the fact remains that its still a drug and can have adverse effects if misused \cite{misuse}.
Using caffeine effectively is a difficult skill to master as it would require logging the caffeine content of each beverage in addition to when it was consumed and then performing some complex calculations to work out the optimum time for the next dosage.
Unfortunately even performing this complex process would not necessarily be enough to manage caffeine levels effectively as they would not take into account daily tasks where obtaining caffeine was infeasible such as in the middle of a lecture. 
Opticaff aims to solve this problem by monitoring a users caffeine intake in addition to their timetabled activities so that consumption suggestions work in harmony with the users routine.
In addition to this it also points the user to the closest place to purchase their desired beverage so that their detour is minimal.  
  
\subsection{Project Audience}
The initial target audience of Opticaff are people who visit the University of Southampton Campus on a regular basis. 
The shops and vending machines that are suggested by Opticaff are all either University of Southampton affiliated or in close proximity to the campus and are therefore regularly accessible to frequent visitors of the University.

\subsection{Project Goals}
The goals of this project are to produce a prototype with the following functionality: 

\begin{enumerate}
	\item{Obtain and use linked data regarding caffeine sources in and around the University of Southampton.}
	\item{Integrate and use University of Southampton timetable data using Sussed}
	\item{Monitor users caffeine intake and predict the optimum time for the next consumption.}
	\item{Have the notion of a leaderboard to rank users ``productvity" based on how well they maintain their caffeine level within the optimum range.}
	\item{Approximate the users position and locate the closest appropriate caffeine sources to them.}
\end{enumerate}

\subsection{Project Scope}
This is a prototype application to showcase the main features and to illustrate what the full application will look like. 
The prototype produced will implement the key features of consuming caffeine, calculations based on a users timetable, directions to the nearest appropriate caffeine source, and ranking on the leaderboard. 
This application will use researched averages for the caffeine content of drinks as opposed to storing individual data for each one and will make assumptions for the average man and woman in terms of optimum caffeine intake. 
In addition the application will be able to add a users calendar data for the aforementioned calculations. 
